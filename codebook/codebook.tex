\documentclass[a4paper,10pt,twocolumn,oneside]{article}
\setlength{\columnsep}{10pt}                                                                    %兩欄模式的間距
\setlength{\columnseprule}{0pt}                                                                %兩欄模式間格線粗細

\usepackage{amsthm}								%定義,例題
\usepackage{amssymb}
%\usepackage[margin=2cm]{geometry}
\usepackage{fontspec}								%設定字體
\usepackage{color}
\usepackage[x11names]{xcolor}
\usepackage{listings}								%顯示code用的
%\usepackage[Glenn]{fncychap}						%排版,頁面模板
\usepackage{fancyhdr}								%設定頁首頁尾
\usepackage{graphicx}								%Graphic
\usepackage{enumerate}
\usepackage{titlesec}
\usepackage{amsmath}
\usepackage[CheckSingle, CJKmath]{xeCJK}
% \usepackage{CJKulem}

%\usepackage[T1]{fontenc}
\usepackage{amsmath, courier, listings, fancyhdr, graphicx}
\topmargin=0pt
\headsep=5pt
\textheight=780pt
\footskip=0pt
\voffset=-40pt
\textwidth=545pt
\marginparsep=0pt
\marginparwidth=0pt
\marginparpush=0pt
\oddsidemargin=0pt
\evensidemargin=0pt
\hoffset=-42pt

%\renewcommand\listfigurename{圖目錄}
%\renewcommand\listtablename{表目錄} 

%%%%%%%%%%%%%%%%%%%%%%%%%%%%%

\setmainfont{FreeMono}				%主要字型
\setmonofont{Monaco}				%主要字型
\setCJKmainfont{FreeMono}
%\setCJKmainfont{FreeMono}			%中文字型
%\setmainfont{sourcecodepro}
\XeTeXlinebreaklocale "zh"						%中文自動換行
\XeTeXlinebreakskip = 0pt plus 1pt				%設定段落之間的距離
\setcounter{secnumdepth}{3}						%目錄顯示第三層

%%%%%%%%%%%%%%%%%%%%%%%%%%%%%
\makeatletter
\lst@CCPutMacro\lst@ProcessOther {"2D}{\lst@ttfamily{-{}}{-{}}}
\@empty\z@\@empty
\makeatother
\lstset{											% Code顯示
language=C++,										% the language of the code
basicstyle=\footnotesize\ttfamily, 						% the size of the fonts that are used for the code
%numbers=left,										% where to put the line-numbers
numberstyle=\footnotesize,						% the size of the fonts that are used for the line-numbers
stepnumber=1,										% the step between two line-numbers. If it's 1, each line  will be numbered
numbersep=5pt,										% how far the line-numbers are from the code
backgroundcolor=\color{white},					% choose the background color. You must add \usepackage{color}
showspaces=false,									% show spaces adding particular underscores
showstringspaces=false,							% underline spaces within strings
showtabs=false,									% show tabs within strings adding particular underscores
frame=false,											% adds a frame around the code
tabsize=2,											% sets default tabsize to 2 spaces
captionpos=b,										% sets the caption-position to bottom
breaklines=true,									% sets automatic line breaking
breakatwhitespace=false,							% sets if automatic breaks should only happen at whitespace
escapeinside={\%*}{*)},							% if you want to add a comment within your code
morekeywords={*},									% if you want to add more keywords to the set
keywordstyle=\bfseries\color{Blue1},
commentstyle=\itshape\color{Red4},
stringstyle=\itshape\color{Green4},
}

%%%%%%%%%%%%%%%%%%%%%%%%%%%%%

\begin{document}
\pagestyle{fancy}
\fancyfoot{}
%\fancyfoot[R]{\includegraphics[width=20pt]{ironwood.jpg}}
\fancyhead[L]{hhhhaura's codebook}
\fancyhead[R]{\thepage}
\renewcommand{\headrulewidth}{0.4pt}
\renewcommand{\contentsname}{Contents} 

\scriptsize
\tableofcontents
%%%%%%%%%%%%%%%%%%%%%%%%%%%%%

\newpage

\section{Basic}
\subsection{Default code}
\lstinputlisting{basic/test.cpp}

\subsection{FasterIO}
\lstinputlisting{basic/fast_io.cpp}

\subsection{Check}
\lstinputlisting{basic/check.sh}

\section{Data Structure}
\subsection{Disjoint set -- Path Compression}
\lstinputlisting{data_structure/Disjoint-set-path_compression.cpp}

\subsection{Disjoint set -- Undo}
\lstinputlisting{data_structure/Disjoin-set-undo.cpp}

\subsection{Sparse Table}
\lstinputlisting{data_structure/Sparse_table.cpp}


\subsection{Persistent Segment Tree}
\lstinputlisting{data_structure/persistent_segment_tree.cpp}

\subsection{Li Chao Tree}
\lstinputlisting{data_structure/Li_chao.cpp}

\subsection{Treap}
\lstinputlisting{data_structure/Treap.cpp}

\section{Graph}

\subsection{D and L}
\lstinputlisting{graph/DL.cpp}

\subsection{Articulation Point}
\lstinputlisting{graph/AP.cpp}

\subsection{Bridge}
\lstinputlisting{graph/Bridge.cpp}

\subsection{Biconnected Component}
\lstinputlisting{graph/BCC.cpp}

\subsection{Strongly Connected Component}
\lstinputlisting{graph/SCC.cpp}

\subsection{Cactus}
\lstinputlisting{graph/cactus.cpp}

\subsection{Block-Cut Tree}
\lstinputlisting{graph/Block_cut_tree.cpp}

\subsection{Centroid Decomposition}
\lstinputlisting{graph/Centroid_decomposition.cpp}

\subsection{Dinic}
\lstinputlisting{graph/Dinic.cpp}

\subsection{Eular}
\lstinputlisting{graph/Eular.cpp}

\subsection{Heavy-Light Decomposition}
\lstinputlisting{graph/HLD.cpp}

\subsection{Hungary}
\lstinputlisting{graph/Hungary.cpp}

\subsection{Jellyfish}
\lstinputlisting{graph/Jellyfish.cpp}

\subsection{KM algo}
\lstinputlisting{graph/KM_algo_slack.cpp}

\subsection{Prim}
\lstinputlisting{graph/Prim.cpp}

\subsection{Smallest Mean Cycle}
\lstinputlisting{graph/smallest_mean_cycle.cpp}

\section{Math}

\subsection{FFT and NTT}
\lstinputlisting{math/FFT.cpp}

\subsection{Chinese Remainder Theorum}
\lstinputlisting{math/Chinese_remainder.cpp}

\subsection{Discrete Log}
\lstinputlisting{math/discreat_log.cpp}

\subsection{EXgcd}
\lstinputlisting{math/Exgcd.cpp}

\subsection{Gauss Elimination}
\lstinputlisting{math/guass.cpp}

\subsection{Linear Basis}
\lstinputlisting{math/linear_base.cpp}

\subsection{Miller Rabin}
\lstinputlisting{math/Miller-rabin.cpp}

\subsection{Mobius Transform}
\lstinputlisting{math/Mobius.cpp}

\section{String}

\section{Geometry}

\section{Others}

\end{document}
